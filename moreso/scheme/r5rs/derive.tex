\section{Derived expression types}
\label{derivedsection}

This section gives macro definitions for the derived expression types in
terms of the primitive expression types (literal, variable, call, {\cf lambda},
{\cf if}, {\cf set!}).  See section~\ref{proceduresection} for a possible
definition of {\cf delay}.

\begin{scheme}
(define-syntax \ide{cond}
  (syntax-rules (else =>)
    ((cond (else result1 result2 ...))
     (begin result1 result2 ...))
    ((cond (test => result))
     (let ((temp test))
       (if temp (result temp))))
    ((cond (test => result) clause1 clause2 ...)
     (let ((temp test))
       (if temp
           (result temp)
           (cond clause1 clause2 ...))))
    ((cond (test)) test)
    ((cond (test) clause1 clause2 ...)
     (let ((temp test))
       (if temp
           temp
           (cond clause1 clause2 ...))))
    ((cond (test result1 result2 ...))
     (if test (begin result1 result2 ...)))
    ((cond (test result1 result2 ...)
           clause1 clause2 ...)
     (if test
         (begin result1 result2 ...)
         (cond clause1 clause2 ...)))))
\end{scheme}

\begin{scheme}
(define-syntax \ide{case}
  (syntax-rules (else)
    ((case (key ...)
       clauses ...)
     (let ((atom-key (key ...)))
       (case atom-key clauses ...)))
    ((case key
       (else result1 result2 ...))
     (begin result1 result2 ...))
    ((case key
       ((atoms ...) result1 result2 ...))
     (if (memv key '(atoms ...))
         (begin result1 result2 ...)))
    ((case key
       ((atoms ...) result1 result2 ...)
       clause clauses ...)
     (if (memv key '(atoms ...))
         (begin result1 result2 ...)
         (case key clause clauses ...)))))
\end{scheme}

\begin{scheme}
(define-syntax \ide{and}
  (syntax-rules ()
    ((and) \sharpfoo{t})
    ((and test) test)
    ((and test1 test2 ...)
     (if test1 (and test2 ...) \sharpfoo{f}))))
\end{scheme}

\begin{scheme}
(define-syntax \ide{or}
  (syntax-rules ()
    ((or) \sharpfoo{f})
    ((or test) test)
    ((or test1 test2 ...)
     (let ((x test1))
       (if x x (or test2 ...))))))
\end{scheme}

\begin{scheme}
(define-syntax \ide{let}
  (syntax-rules ()
    ((let ((name val) ...) body1 body2 ...)
     ((lambda (name ...) body1 body2 ...)
      val ...))
    ((let tag ((name val) ...) body1 body2 ...)
     ((letrec ((tag (lambda (name ...)
                      body1 body2 ...)))
        tag)
      val ...))))
\end{scheme}

\begin{scheme}
(define-syntax \ide{let*}
  (syntax-rules ()
    ((let* () body1 body2 ...)
     (let () body1 body2 ...))
    ((let* ((name1 val1) (name2 val2) ...)
       body1 body2 ...)
     (let ((name1 val1))
       (let* ((name2 val2) ...)
         body1 body2 ...)))))
\end{scheme}

The following {\cf letrec} macro uses the symbol {\cf <undefined>}
in place of an expression which returns something that when stored in
a location makes it an error to try to obtain the value stored in the
location (no such expression is defined in Scheme).
A trick is used to generate the temporary names needed to avoid
specifying the order in which the values are evaluated.
This could also be accomplished by using an auxiliary macro.

\begin{scheme}
(define-syntax \ide{letrec}
  (syntax-rules ()
    ((letrec ((var1 init1) ...) body ...)
     (letrec "generate\_temp\_names"
       (var1 ...)
       ()
       ((var1 init1) ...)
       body ...))
    ((letrec "generate\_temp\_names"
       ()
       (temp1 ...)
       ((var1 init1) ...)
       body ...)
     (let ((var1 <undefined>) ...)
       (let ((temp1 init1) ...)
         (set! var1 temp1)
         ...
         body ...)))
    ((letrec "generate\_temp\_names"
       (x y ...)
       (temp ...)
       ((var1 init1) ...)
       body ...)
     (letrec "generate\_temp\_names"
       (y ...)
       (newtemp temp ...)
       ((var1 init1) ...)
       body ...))))
\end{scheme}

\begin{scheme}
(define-syntax \ide{begin}
  (syntax-rules ()
    ((begin exp ...)
     ((lambda () exp ...)))))
\end{scheme}

The following alternative expansion for {\cf begin} does not make use of
the ability to write more than one expression in the body of a lambda
expression.  In any case, note that these rules apply only if the body
of the {\cf begin} contains no definitions.

\begin{scheme}
(define-syntax begin
  (syntax-rules ()
    ((begin exp)
     exp)
    ((begin exp1 exp2 ...)
     (let ((x exp1))
       (begin exp2 ...)))))
\end{scheme}

The following definition
of {\cf do} uses a trick to expand the variable clauses.
As with {\cf letrec} above, an auxiliary macro would also work.
The expression {\cf (if \#f \#f)} is used to obtain an unspecific
value.

\begin{scheme}
(define-syntax \ide{do}
  (syntax-rules ()
    ((do ((var init step ...) ...)
         (test expr ...)
         command ...)
     (letrec
       ((loop
         (lambda (var ...)
           (if test
               (begin
                 (if \#f \#f)
                 expr ...)
               (begin
                 command
                 ...
                 (loop (do "step" var step ...)
                       ...))))))
       (loop init ...)))
    ((do "step" x)
     x)
    ((do "step" x y)
     y)))
\end{scheme}

% `a                =  Q_1[a]
% `(a b c ... . z)  =  `(a . (b c ...))
% `(a . b)          =  (append Q*_0[a] `b)
% `(a)              =  Q*_0[a]
% Q*_0[a]           =  (list 'a)
% Q*_0[,a]          =  (list a)
% Q*_0[,@a]         =  a
% Q*_0[`a]          =  (list 'quasiquote Q*_1[a])
% `#(a b ...)       =  (list->vector `(a b ...))
%  ugh.
