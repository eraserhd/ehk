\section{Removal of the Distinct String Type}
\subsection{Rationale}

R5RS specifies strings and vectors as disjoint types.  This necessitates
distinct procedures for operating on each type when the procedures would
otherwise have the same meaning (e.g. {\tt string-length} and
{\tt vector-length}, {\tt string->list} and {\tt vector->list}).

Since the domains of these functions do not overlap, one solution would be to
replace the duplicated, type-specific routines with singular, general ones;
however, removing the string type itself is possible without much trouble.

\subsubsection{\tt string?}

{\tt string?} now returns \schtrue when {\tt vector?} would return \schtrue
and {\tt char?} would return \schtrue for every element of the vector.

\begin{scheme}
(string? '\sharpsign())                    \ev \schtrue
(string? (vector \sharpsign\textbackslash{}c))           \ev \schtrue
(string? (vector \sharpsign\textbackslash{}c \schfalse)) \ev \schfalse
\end{scheme}

\subsection{Quoted Literals}

The quoted literal string syntax is kept, except that it now represents a vector
of characters.  That is,

\begin{scheme}
(vector? "x") \ev \schtrue
(string? "x") \ev \schtrue
\end{scheme}


{\tt display}

{\tt write} now produces the quoted literal in the cases where {\tt string?}
would return \schtrue.

\begin{enumerate}
  \item string-length, string-set!, string-ref, string->list, make-string,
	string, list->string, string-fill! are all removed in favor of their
	vector counterparts.
  \item substring is renamed section and works on any kind of vector.
  \item string-append and string-copy become vector-append and vector-copy.
\end{enumerate}

This means that the following procedures are not touched

% vim:set ft=tex:
