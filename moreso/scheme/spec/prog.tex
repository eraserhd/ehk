\chapter{Program structure}
\label{programchapter}

\section{Programs}

A Scheme program consists of a sequence of expressions, definitions,
and syntax definitions.
Expressions are described in chapter~\ref{expressionchapter};
definitions and syntax definitions are the subject of the rest of the
present chapter.

Programs are typically stored in files or entered interactively to a
running Scheme system, although other paradigms are possible;
questions of user interface lie outside the scope of this report.
(Indeed, Scheme would still be useful as a notation for expressing
computational methods even in the absence of a mechanical
implementation.)

Definitions and syntax definitions occurring at the top level of a program
can be interpreted
declaratively.
They cause bindings to be created in the top level
environment or modify the value of existing top-level bindings.
Expressions occurring at the top level of a program are
interpreted imperatively; they are executed in order when the program is
invoked or loaded, and typically perform some kind of initialization.

At the top level of a program {\tt(begin \hyperi{form} \dotsfoo)} is
equivalent to the sequence of expressions, definitions, and syntax definitions
that form the body of the \ide{begin}.

\todo{Cromarty, etc.: disclaimer about top level?}

\section{Definitions}
\label{defines}

Definitions are valid in some, but not all, contexts where expressions
are allowed.  They are valid only at the top level of a \hyper{program}
and at the beginning of a \hyper{body}.
\mainindex{definition}

A definition should have one of the following forms:\mainschindex{define}

\begin{itemize}

\item{\tt(define \hyper{variable} \hyper{expression})}

\item{\tt(define (\hyper{variable} \hyper{formals}) \hyper{body})}

\hyper{Formals} should be either a
sequence of zero or more variables, or a sequence of one or more
variables followed by a space-delimited period and another variable (as
in a lambda expression).  This form is equivalent to
\begin{scheme}
(define \hyper{variable}
  (lambda (\hyper{formals}) \hyper{body}))\rm.%
\end{scheme}

\item{\tt(define (\hyper{variable} .\ \hyper{formal}) \hyper{body})}

\hyper{Formal} should be a single
variable.  This form is equivalent to
\begin{scheme}
(define \hyper{variable}
  (lambda \hyper{formal} \hyper{body}))\rm.%
\end{scheme}

\end{itemize}

\subsection{Top level definitions}

At the top level of a program, a definition
\begin{scheme}
(define \hyper{variable} \hyper{expression})%
\end{scheme}
has essentially the same effect as the assignment expression
\begin{scheme}
(\ide{set!}\ \hyper{variable} \hyper{expression})%
\end{scheme}
if \hyper{variable} is bound.  If \hyper{variable} is not bound,
however, then the definition will bind \hyper{variable} to a new
location before performing the assignment, whereas it would be an error
to perform a {\cf set!}\ on an unbound\index{unbound} variable.

\begin{scheme}
(define add3
  (lambda (x) (+ x 3)))
(add3 3)                            \ev  6
(define first car)
(first '(1 2))                      \ev  1%
\end{scheme}

Some implementations of Scheme use an initial environment in
which all possible variables are bound to locations, most of
which contain undefined values.  Top level definitions in
such an implementation are truly equivalent to assignments.

\todo{Rozas: equal time for opposition semantics?}


\subsection{Internal definitions}
\label{internaldefines}

Definitions may occur at the
beginning of a \hyper{body} (that is, the body of a \ide{lambda},
\ide{let}, \ide{let*}, \ide{letrec}, \ide{let-syntax}, or \ide{letrec-syntax}
expression or that of a definition of an appropriate form).
Such definitions are known as {\em internal definitions} \mainindex{internal
definition} as opposed to the top level definitions described above.
The variable defined by an internal definition is local to the
\hyper{body}.  That is, \hyper{variable} is bound rather than assigned,
and the region of the binding is the entire \hyper{body}.  For example,

\begin{scheme}
(let ((x 5))
  (define foo (lambda (y) (bar x y)))
  (define bar (lambda (a b) (+ (* a b) a)))
  (foo (+ x 3)))                \ev  45%
\end{scheme}

A \hyper{body} containing internal definitions can always be converted
into a completely equivalent {\cf letrec} expression.  For example, the
{\cf let} expression in the above example is equivalent to

\begin{scheme}
(let ((x 5))
  (letrec ((foo (lambda (y) (bar x y)))
           (bar (lambda (a b) (+ (* a b) a))))
    (foo (+ x 3))))%
\end{scheme}

Just as for the equivalent {\cf letrec} expression, it must be
possible to evaluate each \hyper{expression} of every internal
definition in a \hyper{body} without assigning or referring to
the value of any \hyper{variable} being defined.

Wherever an internal definition may occur
{\tt(begin \hyperi{definition} \dotsfoo)}
is equivalent to the sequence of definitions
that form the body of the \ide{begin}.

\section{Syntax definitions}

Syntax definitions are valid only at the top level of a \hyper{program}.
\mainindex{syntax definition}
They have the following form:\mainschindex{define-syntax}

{\tt(define-syntax \hyper{keyword} \hyper{transformer spec})}

\hyper{Keyword} is an identifier, and
the \hyper{transformer spec} should be an instance of \ide{syntax-rules}.
The top-level syntactic environment is extended by binding the
\hyper{keyword} to the specified transformer.

There is no {\cf define-syntax} analogue of internal definitions.

%[Rationale flushed because it may or may not be true and isn't the
% real rationale anyway. -RK]
%\begin{rationale}
%As discussed below, the syntax and scope rules for syntax definitions
%can give rise to syntactic ambiguities when syntactic keywords are
%shadowed.
%Further ambiguities would arise if {\cf define-syntax}
%were permitted at the beginning of a \meta{body}, with scope
%rules analogous to those for internal definitions.
%\end{rationale}

% It is an error for a program to contain more than one top-level
% \meta{definition} or \meta{syntax definition} of any identifier.
%
% [I flushed this because it isn't an error for a program to
% contain more than one top-level definition of an identifier,
% and I didn't want to introduce any gratuitous incompatibilities
% with the existing Scheme language. -- Will]

Although macros may expand into definitions and syntax definitions in
any context that permits them, it is an error for a definition or syntax
definition to shadow a syntactic keyword whose meaning is needed to
determine whether some form in the group of forms that contains the
shadowing definition is in fact a definition, or, for internal definitions,
is needed to determine the boundary between the group and the expressions
that follow the group.  For example, the following are errors:

\begin{scheme}
(define define 3)

(begin (define begin list))

(let-syntax
  ((foo (syntax-rules ()
          ((foo (proc args ...) body ...)
           (define proc
             (lambda (args ...)
               body ...))))))
  (let ((x 3))
    (foo (plus x y) (+ x y))
    (define foo x)
    (plus foo x)))
\end{scheme}


